\documentclass[11pt]{article}
\usepackage[utf8]{inputenc}
\usepackage[left=3cm, right=4cm, top=3cm, bottom=4cm]{geometry}
\usepackage{pdfsync}
\usepackage{enumitem}
\usepackage{hyperref}


% Table spacing
%  1 is the default, change whatever you need
\def\arraystretch{1}
\linespread{1.2}
\setlength{\tabcolsep}{.8em}

\title{Curriculum vitae}
\author{Thomas Dybdahl Ahle}
\date{May 2020}
\begin{document}
\maketitle


\paragraph{Education}
\begin{itemize}
   \item[]
      June 2019
      Doctor of Philosophy,
      IT University of Copenhagen.
   \item[]
      2019
      Master of Arts in Computer Science,
      University of Oxford.
   \item[]
      2017
      Master of Science,
      IT University of Copenhagen, University of Copenhagen.
   \item[]
      2013
      Bachelor of Arts in Computer Science,
      University of Oxford.
\end{itemize}

\paragraph{Publications}
\begin{itemize}
   \item[]
   ``Subsets and Supermajorities: Optimal Hashing-based Set Similarity Search''.
   By
         Thomas Dybdahl Ahle       -- Submitted,
    2019.
   \item[]
   ``Oblivious Sketching of High-Degree Polynomial Kernels''.
   By
         Thomas Dybdahl Ahle,       M Kapralov,       J Knudsen,       R Pagh,       A Velingker,       D Woodruff,       A Zandieh       at
       ACM-SIAM Symposium on Discrete Algorithms (SODA),
    2019.
   \item[]
   ``Optimal Las Vegas Locality Sensitive Data Structures''.
   By
         Thomas Dybdahl Ahle       at
       IEEE Symposium on Foundations of Computer Science (FOCS),
    2017.
   \item[]
   ``Parameter-free Locality Sensitive Hashing for Spherical Range Reporting''.
   By
         Thomas Dybdahl Ahle,       M Aumüller,       R Pagh       at
       ACM-SIAM Symposium on Discrete Algorithms (SODA),
    2017.
   \item[]
   ``On the Complexity of Inner Product Similarity Join''.
   By
         Thomas Dybdahl Ahle,       R Pagh,       I Razenshteyn,       F Silvestri       at
       ACM Symposium on Principles of Database Systems (PODS),
    2016.
\end{itemize}

\paragraph{Awards and Scholarships}
\begin{itemize}
   \item[]
   \emph{Research Travel Award,
            Stibo-Foundation,
      2016}.
   \\
   Given to just two Danish students a year, to collaborate in research abroad.
   \item[]
   \emph{Northwestern Europe Regional Programming Contest,
               1st,
            Association for Computing Machinery,
      2014}.
   \\
   With my team Lambdabamserne, becoming the first ever Danish team to qualify for the ACM wold finals.
   \item[]
   \emph{Danish National Programming Champion,
               1st,
            Netcompany,
      2013, 2014}.
   \\
   Algorithm competition known as "DM i Programmering" 
   \item[]
   \emph{Oxford Computer Science Competition,
               1st,
            University of Oxford,
      2013}.
   \\
   For my Numberlink solving software, giving the first fixed parameter polynomial algorithm for the problem.
   \item[]
   \emph{Demyship,
            Magdalen College,
      2010, 2011}.
   \\
   A historic scholarship awarded to the top students each year.
   \item[]
   \emph{Les Trophées du Libre,
               1st,
            Free Software Foundation Europe,
      2007}.
   \\
   For my work on the PyChess free software chess suite.
\end{itemize}

\paragraph{Industry and Employment}
\begin{itemize}
      \item[]
      \emph{Chief Machine Learning Officer at SupWiz, 2017 - 2018}.
      \\
      I co-founded an NLP start-up with academics from University of Copenhagen.
               At SupWiz I lead a team of four in developing our chatbot software and putting it into production at 3 of the largest Danish IT companies. (Now many more.)
               In 2019 the chatbot won the most prestigious prize given by Innovation Fund Denmark.
               I was also responsible for our hiring efforts, interviewing dozens and employing 4 engineers over a 5 month period.
               
      \item[]
      \emph{Teaching at IT University of Copenhagen, 2015 - 2019}.
      \\
      In 2019 I co-designed and taught the Parallel and Concurrent Programming course to 140 master students.
               Earlier years I assisted in various algorithms design classes.
            
      \item[]
      \emph{Teaching at University of Copenhagen, 2014}.
      \\
      I assisted in teaching algorithms to more than 200 bachelor students.
\end{itemize}

\paragraph{Open Source Projects}
\begin{itemize}
      \item[]
      \emph{Project Owner at PyChess, 2006 - current}.
      \\
      Developed the most used chess client and engine for the Linux desktop. Currently the 7th most used interface on the Free Internet Chess Server. Translated to more than 35 languages. I lead a team of 4-8 developers and designers. In 2009 we won Les Trophées du Libre in Paris. The project is under the Gnu Public License and has been used by people all over the world for research projects and other experiments.
\end{itemize}


\paragraph{Media}
\begin{itemize}
   \item[]
      \emph{"The Stibo-Foundation supports IT-talents", Stibo, August 2016.}
      The announcement of my winning the Stibo Travel grant.
   \item[]
      \emph{Bidwell, Jonni.
         "Python: Sunfish chess engine", Linux Format, January 2016.}
      Article about my Sunfish chess software.
   \item[]
      \emph{"The National Team at the Programming World Cup", Computerworld, June 2015.}
      Coverage of my teams participation in the ICPC World Finals.
   \item[]
      \emph{Elkær, Mads.
         "Denmark's Three Greatest Programmers", Computerworld, October 2013.}
      
\end{itemize}

\paragraph{Contact}
\begin{itemize}
   \item[]
      Email: \href{mailto:thdy@itu.dk}{thdy@itu.dk}.
   \item[]
      Website: \href{http://www.thomasahle.com}{\underline{thomasahle.com}}
   \item[]
      \href{https://dblp1.uni-trier.de/pers/hd/a/Ahle:Thomas_D=}{\underline{DBLP List of papers}}
   \item[]
      \href{https://scholar.google.dk/citations?user=aRiVoYgAAAAJ}{\underline{Google Scholar List of Papers}}
   \item[]
      Linkedin: \href{https://www.linkedin.com/in/thomasahle/}{\underline{linkedin.com/in/thomasahle}}
   \item[]
      Github: \href{https://github.com/thomasahle}{\underline{github.com/thomasahle}}
\end{itemize}

\end{document}
